\chapter*{}
%\thispagestyle{empty}
%\cleardoublepage

%\thispagestyle{empty}

\input{portada/portada_2}



\cleardoublepage
\thispagestyle{empty}

\begin{center}
{\large\bfseries INDI Web Cliente: Desarrollo de un prototipo de cliente Web INDI para el control de instrumental astronómico}\\
\end{center}
\begin{center}
Pablo Torrecillas Ortega\\
\end{center}

%\vspace{0.7cm}
\noindent{\textbf{Palabras clave}: INDI, Software Libre, Cliente Web, Internet, Astronomía, Control, Prototipo, Navegador}\\

\vspace{0.7cm}
\noindent{\textbf{Resumen}}\\

Este proyecto ha sido realizado ante la necesidad de conseguir diferentes herramientas para el control de instrumental astronómico y facilitar el trabajo tanto a profesionales como a amateurs en el campo de la astronomía. Una vez analizado el estado del arte y los medios que se utilizan en este momento, se ha desarrollado un proyecto encaminado a conseguir los objetivos planteados a través de un prototipo de cliente web viniendo a completar de esta forma las diferentes aplicaciones que ya existen.\\

El objetivo del proyecto es poder realizar la función control de un observatorio astronómico situado en cualquier lugar del mundo con conexión a Internet.\\
Las tareas que se pueden realizar son las siguientes: recepción y procesamiento de datos de diferente naturaleza, envío de los mismos y cambio de parámetros y valores. Este trabajo viene a ser la base para otras funciones posteriores que se pueden desarrollar a partir del prototipo de cliente web realizado.\\

Para ello, se ha utilizado como punto de partida el protocolo INDI que ya se encuentra desarrollado. INDI es, a grandes rasgos, un protocolo que facilita el control en el tiempo y en el espacio, así como la adquisición de datos y el intercambio entre diferentes dispositivos hardware y sus interfaces software.\\
La utilización de este protocolo, lleva aparejado una enorme ventaja y es el poder emplear diferentes dispositivos obteniendo resultados con el mismo beneficio, independientemente de la naturaleza del dispositivo utilizado, abriéndose así un abanico de posibilidades ante el problema que se podría plantear si se tuviera que utilizar un dispositivo concreto y único, pues de esa forma se acotarían enormemente las posibilidades si no se cuenta con el dispositivo en cuestión y por tanto no se podría llevar a cabo la función control. Cabe destacar que este prototipo de cliente es multiplataforma y por tanto es capaz de ejecutarse en cualquier máquina que disponga de navegador web.\\

La aportación de este proyecto, basado en un prototipo de cliente web, es una nueva e importante posibilidad  que viene a unirse a las ya existentes, complementando la forma de obtener resultados utilizando otras herramientas. \\

Dado que la difusión de Internet se hace ya de forma cotidiana y es fácil encontrar un punto de acceso al mismo, prácticamente en cualquier punto del planeta el astrónomo tanto amateur como  profesional, solo necesitaría un navegador web y la conexión a Internet para poder trabajar y controlar todos los aspectos astronómicos que desee.\\

Este proyecto aporta una innovación de forma sencilla, rápida y poco costosa en la forma de trabajar en la obtención de datos y su posterior manipulación en el campo de la astronomía.\\

Hay que hacer justa mención al software libre y su filosofía pues han sido pilares básicos en el trabajo que hoy se presenta, ya que se han utilizado todas las herramientas de dicho software para el desarrollo del prototipo del cliente web.\\

Como conclusión, cabe destacar que este proyecto sienta las bases para conseguir llegar a ser un cliente web como producto final utilizando el trabajo desarrollado y que se iniciaba creando un prototipo de cliente web basado en INDI. \\


\cleardoublepage


\thispagestyle{empty}


\begin{center}
{\large\bfseries Project Title: Project Subtitle}\\
\end{center}
\begin{center}
First name, Family name (student)\\
\end{center}

%\vspace{0.7cm}
\noindent{\textbf{Keywords}: Keyword1, Keyword2, Keyword3, ....}\\

\vspace{0.7cm}
\noindent{\textbf{Abstract}}\\

Write here the abstract in English.

\chapter*{}
\thispagestyle{empty}

\noindent\rule[-1ex]{\textwidth}{2pt}\\[4.5ex]

Yo, \textbf{Pablo Torrecillas Ortega}, alumno de la titulación Grado de Ingeniería Informática de  \textbf{Escuela Técnica Superior
de Ingenierías Informática y de Telecomunicación de la Universidad de Granada}, con DNI XXXXXXXXX, autorizo la
ubicación de la siguiente copia de mi Trabajo Fin de Grado en la biblioteca del centro para que pueda ser
consultada por las personas que lo deseen.

\vspace{6cm}

\noindent Fdo: Pablo Torrecillas Ortega

\vspace{2cm}

\begin{flushright}
Granada a 11 de diciembre de 2015.
\end{flushright}


\chapter*{}
\thispagestyle{empty}

\noindent\rule[-1ex]{\textwidth}{2pt}\\[4.5ex]

D. \textbf{Sergio Alonso Burgos}, Profesor del Departamento de Lenguajes y Sistemas Informáticos de la Universidad de Granada.

\vspace{0.5cm}

\textbf{Informa:}

\vspace{0.5cm}

Que el presente trabajo, titulado \textit{\textbf{INDI Web Client, Desarrollo de un prototipo de cliente Web INDI para el control de instrumental astronómico}},
ha sido realizado bajo su supervisión por \textbf{Pablo Torrecillas Ortega}, y autorizamos la defensa de dicho trabajo ante el tribunal
que corresponda.

\vspace{0.5cm}

Y para que conste, expiden y firman el presente informe en Granada a 11 de diciembre de 2015.

\vspace{1cm}

\textbf{Tutor:}

\vspace{5cm}

\noindent \textbf{Prof. Dr. Sergio Alonso Burgos }

\chapter*{Agradecimientos}
\thispagestyle{empty}

       \vspace{1cm}


Poner aquí agradecimientos...
