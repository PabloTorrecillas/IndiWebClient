\chapter*{}
%\thispagestyle{empty}
%\cleardoublepage

%\thispagestyle{empty}

\input{portada/portada_2}



\cleardoublepage
\thispagestyle{empty}

\begin{center}
{\large\bfseries INDI Web Client: Desarrollo de un prototipo de cliente Web INDI para el control de instrumental astronómico}\\
\end{center}
\begin{center}
Pablo Torrecillas Ortega\\
\end{center}

%\vspace{0.7cm}
\noindent{\textbf{Palabras clave}: INDI, \textit{Software Libre}, Cliente Web, Internet, Astronomía, Control, Prototipo, Navegador}\\

\vspace{0.7cm}
\noindent{\textbf{Resumen}}\\

Este proyecto ha sido realizado ante la necesidad de conseguir diferentes herramientas para el control de instrumental astronómico y facilitar el trabajo tanto a profesionales como a \textit{amateurs} en el campo de la astronomía. Una vez analizado el estado del arte y los medios que se utilizan en este momento, se ha desarrollado un proyecto encaminado a construir un prototipo de cliente Web viniendo a completar de esta forma las diferentes aplicaciones que ya existen para gestionar un observatorio remoto.\\

El objetivo del proyecto es poder realizar la función control de un observatorio astronómico situado en cualquier lugar del mundo con conexión a Internet y un navegador Web estándar.\\

Las tareas que se pueden realizar son las siguientes: recepción y procesamiento de datos de diferente naturaleza, envío de los mismos y cambio de parámetros y valores. Este trabajo viene a ser la base para otras funciones posteriores que se pueden desarrollar a partir del prototipo de cliente Web realizado.\\

Para ello, se ha utilizado como punto de partida el protocolo INDI que ya se encuentra desarrollado. INDI es, a grandes rasgos, un protocolo que facilita el control de un observatorio, así como la adquisición de datos y el intercambio entre diferentes dispositivos \textit{hardware} y sus interfaces \textit{software}.\\

La utilización de este protocolo lleva aparejada una enorme ventaja. Esta ventaja es el poder emplear diferentes dispositivos obteniendo resultados con el mismo beneficio, independientemente de la naturaleza del dispositivo utilizado.\\
Se abre por tanto, un abanico de posibilidades ante el problema que se podría plantear si se tuviera que utilizar un dispositivo concreto y único. De esa forma se acotarían enormemente las posibilidades si no se cuenta con el dispositivo en cuestión y por tanto no se podría llevar a cabo la función control. Cabe destacar que este prototipo de cliente es multiplataforma y por tanto es capaz de ejecutarse en cualquier máquina que disponga de navegador Web.\\

La aportación de este proyecto, basado en un prototipo de cliente Web, es una nueva e importante posibilidad  que viene a unirse a las ya existentes, complementando la forma de obtener resultados utilizando otras herramientas. \\

Dado que la difusión de Internet se hace ya de forma cotidiana y es fácil encontrar un punto de acceso al mismo, prácticamente en cualquier punto del planeta el astrónomo tanto \textit{amateur} como  profesional, solo necesitaría un navegador Web y la conexión a Internet para poder trabajar y controlar todos los aspectos de su observatorio como desee.\\

Este proyecto aporta una innovación de forma sencilla, rápida y poco costosa en la forma de trabajar en la obtención de datos y su posterior manipulación en el campo de la astronomía.\\

Hay que hacer justa mención al \textit{Software Libre} y su filosofía pues han sido pilares básicos en el trabajo que hoy se presenta, ya que se han utilizado todas las herramientas de dicho \textit{software} para el desarrollo del prototipo del cliente Web.\\

Como conclusión, cabe destacar que este proyecto sienta las bases para conseguir llegar a ser un cliente Web como producto final utilizando el trabajo desarrollado y que se iniciaba creando un prototipo de cliente Web basado en INDI. \\


\cleardoublepage


\thispagestyle{empty}


\begin{center}
{\large\bfseries INDI Web Client: INDI Web Client Prototype development for astronomical instrumental control}\\
\end{center}
\begin{center}
Pablo Torrecillas Ortega\\
\end{center}

%\vspace{0.7cm}
\noindent{\textbf{Keywords}: INDI, Open Source, Web Client, Internet, Astronomy, Control, Prototype, Web Browser}\\

\vspace{0.7cm}
\noindent{\textbf{Abstract}}\\

This project emerges from a pressing need to achieve different tools for the control of astronomical equipment and to simplify the work of both professionals and amateurs in the field of astronomy. Having analyzed the different astronomical softwares and the means used at the moment, has been developed a project designed to achieve the aims stated through a Web client prototype based on INDI, completing in this way the different existing applications.\\

This project emerges from a pressing need to achieve different tools for the control of astronomical equipment and to simplify the work of both professionals and amateurs in the field of astronomy. Having analyzed the different astronomical softwares and the means used at the moment, has been developed a project designed to achieve the aims stated through a Web client prototype based on INDI, completing in this way the different existing applications.\\

The tasks that can be performed are the following: different nature data reception and processing, sending of the same and change of parameters and values. This work is the basis for the subsequent functions that can be developed from the Web client prototype made.\\

In doing so, the protocol INDI has been used as a starting point because it is developed. INDI is, in general, a protocol that facilitates the control in the time and the space, as well as the data acquisition and the exchange between different hardware devices and its software interfaces.\\

The use of this protocol carries a huge advantage and different devices can be used obtaining results with the same benefit, regardless of the nature of the device used. There is a range of possibilities in response to the problem that could raise if an unique and specific device would be used, as it means that the possibilities would be greatly delimited without the concerned  device and therefore the control function could not be performed. It has to be emphasized that this Web client prototype has been developed to be run in any personal computer with Web browser and Internet connection.\\

This project’s contribution, based on a Web client prototype, is a new and important possibility which comes to embrace the existing ones, complementing the way to obtain results using other tools.\\

Considering that the spread of the Internet is made on a daily basis and that it is easy to find a connection point, practically anywhere on the planet both amateur and professional astronomers would only need a Web browser and Internet connection to be able to control and work with every desired astronomical aspects.\\

This project contributes to an innovation in an easy, fast and inexpensive way to the way of working in obtaining data and its subsequent manipulation in the field of astronomy.\\

It’s worth mentioning that the Open Source and its philosophy have been the basic pillars in the work presented today,  since every software tool has been used to the Web client prototype development based on INDI.\\

In conclusion, it’s worth stressing that this project provides the basis for becoming a Web client as final product using the work developed and that it was initiated creating a Web client prototype based on INDI. This project can be important for astronomers as they do not have the need of going to the astronomical observatory.

\chapter*{}
\thispagestyle{empty}

\noindent\rule[-1ex]{\textwidth}{2pt}\\[4.5ex]

Yo, \textbf{Pablo Torrecillas Ortega}, alumno de la titulación Grado de Ingeniería Informática de  \textbf{Escuela Técnica Superior
de Ingenierías Informática y de Telecomunicación de la Universidad de Granada}, con DNI 76439188E, autorizo la
ubicación de la siguiente copia de mi Trabajo Fin de Grado en la biblioteca del centro para que pueda ser
consultada por las personas que lo deseen.

\vspace{1cm}
\includegraphics[width=0.4\textwidth]{./imagenes/firmaPablo}

\noindent Fdo: Pablo Torrecillas Ortega

\vspace{2cm}

\begin{flushright}
Granada a 11 de diciembre de 2015.
\end{flushright}


\chapter*{}
\thispagestyle{empty}

\noindent\rule[-1ex]{\textwidth}{2pt}\\[4.5ex]

D. \textbf{Sergio Alonso Burgos}, Profesor del Departamento de Lenguajes y Sistemas Informáticos de la Universidad de Granada.

\vspace{0.5cm}

\textbf{Informa:}

\vspace{0.5cm}

Que el presente trabajo, titulado \textit{\textbf{INDI Web Client, Desarrollo de un prototipo de cliente Web INDI para el control de instrumental astronómico}},
ha sido realizado bajo su supervisión por \textbf{Pablo Torrecillas Ortega}, y autorizamos la defensa de dicho trabajo ante el tribunal
que corresponda.

\vspace{0.5cm}

Y para que conste, expiden y firman el presente informe en Granada a 11 de diciembre de 2015.

\vspace{1cm}

\textbf{Tutor:}

\vspace{1cm}

\includegraphics[width=0.5\textwidth]{./imagenes/firmaZerjillo3}

\noindent \textbf{Prof. Dr. Sergio Alonso Burgos }

\chapter*{Agradecimientos}
\thispagestyle{empty}

       \vspace{1cm}

A mi madre por toda su dedicación, cariño, esfuerzo y sacrificio para llegar a ser lo que soy.\\

A mi hermano por su apoyo incondicional ya que gracias a él soy mejor persona y mejor informático.\\

A mi novia Nuria por animarme, quererme y apoyarme en todos los momentos, y en especial en los más difíciles.\\

Al resto de mi familia y amigos porque gracias a ellos nunca he perdido la sonrisa y he crecido muchísimo como persona.\\

A mi tutor Sergio Alonso Burgos por todo su empeño, paciencia y confianza depositada en mi.
