\chapter{Conclusiones}
\title{Conclusiones}
\label{cap:Conclusiones}

Una vez que se ha finalizado el desarrollo del proyecto, se evalúa el grado de consecución de los objetivos planteados  al inicio de este trabajo, y se observa que se han alcanzado en su totalidad, cada uno en su justa medida y atendiendo a su amplitud e importancia dentro del nivel que han ocupado en este trabajo. \\

Los objetivos más amplios y que se han conseguido trabajando en profundidad el tema han sido los objetivos principales del prototipo, y se han superado en su integridad, extensamente.
Los objetivos secundarios también se han conseguido en su totalidad. Debo hacer una salvedad respecto a dos de los objetivos secundarios planteados inicialmente, cuales son los objetivos secundarios para “la difusión del prototipo en la página Web de INDI” y el objetivo secundario proyectado para “la divulgación del prototipo en el foro de la Web de INDI”. Estas dos secciones del proyecto se han trabajado someramente porque se ha pensado que ambos apartados se alejaban o no eran imprescindibles en el tema central del proyecto, dejando la posibilidad a otros estudiosos del tema para su desarrollo. Por tanto, se puede decir que ambos objetivos se han conseguido en la medida en que se han trabajado esos apartados.\\

Puntualizado este extremo, determinamos que la evaluación final hay que calificarla como muy positiva, ya que se ha conseguido desarrollar un prototipo de cliente Web haciendo combinación de varias tecnologías y ofreciendo al cliente un amplio abanico de posibilidades a realizar.\\

Por otra parte, cabe destacar en este cliente Web, lo genuino, la originalidad y la innovación que representa su creación y desarrollo, puesto que actualmente no existe un cliente Web capaz de alcanzar lo que si se consigue ejecutar en este prototipo de cliente, ser  capaz de realizar la función control de un observatorio, sea cual sea su localización, siempre que esté conectado a un servidor INDI y desde cualquier punto geográfico del planeta Tierra.\\

Como futuro cliente, debería:
\begin{itemize}
  \item Gestionar varias conexiones a diferentes servidores INDI.
  \item Posibilitar la desconexión de un servidor INDI.
  \item Desarrollar mecanismos de seguridad y privacidad en el cliente.
  \item Modificar la interfaz de usuario para hacerla más atractiva.
\end{itemize}

Por último hay que tener en cuenta un aspecto importante. Todas las pruebas que se han desarrollado en el prototipo de cliente han sido de carácter interno. Ninguna prueba ha sido documentada y todas las que se han realizado han sido para efectuar pequeñas comprobaciones y confirmar que no se habían recibido o enviado datos erróneos.\\

Como conclusión, hay que decir que el Prototipo de Cliente Web basado en INDI es susceptible de ser mejorado en más aspectos ya que está en sus inicios y se pretende conseguir que llegue a ser un cliente Web fiable basado en INDI, a la vez de que sea capaz de realizar, en toda su extensión, la función control de observatorios que estén en cualquier parte del mundo y tengan un servidor INDI.\\
