\chapter{Conclusiones}
\title{Conclusiones}
\label{cap:Conclusiones}

Una vez que se ha finalizado el desarrollo del proyecto, se evalúa el grado de consecución de los objetivos planteados  al inicio de este trabajo, y se observa que se han alcanzado en su totalidad, cada uno en su justa medida y atendiendo a su amplitud e importancia dentro del nivel que han ocupado en este trabajo. \\

Los objetivos más amplios y que se han conseguido trabajando en profundidad el tema han sido los objetivos principales del prototipo, y se han superado en su integridad, extensamente.\\

Los objetivos secundarios se han conseguido casi todos, ya que los objetivos de difusión del proyecto no parecían oportunos realizarlos puesto que el proyecto está en una fase de prototipo y por tanto no está terminado en su totalidad. Por otra parte, una vez que el prototipo pase a ser un cliente web se realizarán dichos objetivos para que todo el que quiera pueda acceder a la información y usar dicho cliente.\\

Puntualizado este extremo, determinamos que la evaluación final hay que calificarla como muy positiva, ya que se ha conseguido desarrollar un prototipo de cliente Web haciendo combinación de varias tecnologías y ofreciendo al cliente un amplio abanico de posibilidades a realizar.\\

Por otra parte, cabe destacar en este cliente Web, lo genuino, la originalidad y la innovación que representa su creación y desarrollo, puesto que actualmente no existe un cliente Web capaz de alcanzar lo que si se consigue ejecutar en este prototipo de cliente, ser  capaz de realizar la función control de un observatorio, sea cual sea su localización, siempre que esté conectado a un servidor INDI y desde cualquier punto geográfico del planeta Tierra.\\

Pese que los objetivos primarios han sido completados en su totalidad, se puede decir que en el futuro se podría mejorar en los siguientes aspectos:
\begin{itemize}
  \item Gestionar varias conexiones a diferentes servidores INDI.
  \item Posibilitar la desconexión de un servidor INDI.
  \item Desarrollar mecanismos de seguridad y privacidad en el cliente.
  \item Modificar la interfaz de usuario para hacerla más atractiva.
\end{itemize}

Como conclusión, hay que decir que el Prototipo de Cliente Web basado en INDI es susceptible de ser mejorado en más aspectos ya que está en sus inicios y se pretende conseguir que llegue a ser un cliente Web fiable basado en INDI, a la vez de que sea capaz de realizar, en toda su extensión, la función control de observatorios que estén en cualquier parte del mundo y tengan un servidor INDI.\\
