\chapter{Objetivos}
\title{Objetivos}
\label{cap:Objetivos}

El objetivo fundamental de este proyecto es desarrollar un prototipo de cliente Web basado en INDI que sea capaz de realizar la función control sobre un observatorio astronómico. Dicho prototipo será multiplataforma y utilizará tecnologías de carácter \textit{Software Libre}. Se pretende además, que en la medida de lo posible se añadan nuevas funcionalidades en un futuro y que se convierta en un cliente Web y no en un prototipo.


\section{Objetivos Principales y Secundarios}
A continuación se detallan los objetivos principales del prototipo del cliente Web:

\begin{itemize}
  \item \textbf{OBJ-1:} Crear un prototipo de cliente que sea capaz de realizar el control de cualquier dispositivo INDI sin apreciar su naturaleza.
  \item \textbf{OBJ-2:} Permitir la conexión con un servidor concreto mediante su IP y su puerto.
  \item \textbf{OBJ-3:} Efectuar la gestión de múltiples dispositivos que estén conectados a un servidor.
  \item \textbf{OBJ-4:} Desarrollar un prototipo de cliente multiplataforma, que permita hacer uso del prototipo del cliente en los principales navegadores que existen en el mercado.
  \item \textbf{OBJ-5:} Utilizar \textit{Software Libre} en la aplicación como herramienta de trabajo.
  \item \textbf{OBJ-6:} Ofrecer la posibilidad de que cualquier desarrollador pueda aportar su capacidad y conocimiento al prototipo.
  \item \textbf{OBJ-7:} Aportar una nueva forma de acceder a los datos de un observatorio astronómico.
  \item \textbf{OBJ-8:} Optimizar el uso de instrumentos de astronomía facilitando una nueva herramienta de obtención de datos.
\end{itemize}

Por otra parte, se busca conseguir los siguientes objetivos secundarios:

\begin{itemize}
  \item \textbf{OBJ-S-1:} Adaptar el prototipo de cliente Web a las propiedades estándares de INDI.
  \item \textbf{OBJ-S-2:} Adecuar el prototipo a los estándares de HTML, CSS, Jquery y JavaScript.
  \item \textbf{OBJ-S-3:} Desarrollar el prototipo de modo que funcione correctamente con el mayor número posible de dispositivos conectados al servidor.
  \item \textbf{OBJ-S-4:} Conseguir que la interfaz sea maximizable y minimizable independientemente del número de dispositivos conectados al servidor.
  \item \textbf{OBJ-S-5:} Difundir el prototipo de cliente Web en la página Web de INDI.
  \item \textbf{OBJ-S-6:} Divulgar el prototipo de cliente Web en el foro de INDI
\end{itemize}


\section{Materias y Herramientas para el Desarrollo}
Este trabajo no se hubiera podido desarrollar sin haber obtenido unos conocimientos básicos y en profundidad en las siguiente materias:
\begin{itemize}
  \item Fundamentos de la Ingeniería del \textit{Software} para realizar el análisis y la ingeniería de requisitos.
  \item Infraestructuras Virtuales para realizar el conjunto de pruebas y simulaciones.
  \item Transmisión de Datos y Redes de Computadores para configurar las conexiones y la red a la hora de hacer uso del prototipo.
  \item Tecnologías Web para poder realizar una Web en HTML y darle un estilo propio con CSS.
  \item Programación Web para poder complementar aún más la materia de Tecnologías Web.
  \item Diseño de Aplicaciones para Internet para desarrollar una aplicación multiplataforma y que funcione en cualquier ordenador con conexión a internet.
\end{itemize}

Por último, hay que hacer mención a los importantes e imprescindibles conocimientos obtenidos en otros campos como:
\begin{itemize}
  \item Astronomía, para entender todo lo que se va buscando en ella.
  \item Equipos astronómicos, para entender cómo funcionan y los resultados que se esperan de ellos.
  \item Websocket, para establecer un canal bidireccional entre el cliente y el servidor.
  \item SublimeText, para poder desarrollar el código fuente del trabajo.
  \item LaTeX, para poder realizar la documentación del proyecto.
  \item GitHub, para realizar un control de versiones del proyecto tanto en la parte de desarrollo del código fuente como en la parte de desarrollo de la documentación.
  \item GanttProject, para realizar los diferentes diagramas de Gantt \cite{DiagramaGantt} usados en la planificación.
\end{itemize}


\section{Alcance de los Objetivos}
El prototipo de cliente Web debe cumplir con todos los objetivos específicos que más arriba exponemos y que nos hemos planteado ante el reto que supone este trabajo. El que así sea es de relevante importancia dado que actualmente no existe ningún cliente Web basado en INDI para controlar un observatorio astronómico.

Así mismo, una vez que se finalice el proyecto, se desea concluir con el prototipo y que pase a ser un cliente Web estable y que los profesionales y estudiosos en la materia que nos ocupa puedan hacer uso y disfrute del mismo.

Esta es una de las razones principales por las que se ha desarrollado este prototipo. Se ha desarrollado bajo \textit{Software Libre} con la intención de que cualquier desarrollador pueda continuar con el proyecto bajo los estándares de \textit{Software Libre}.

Por otra parte, se intentará que dicho prototipo sea publicado en la página Web oficial de INDI para que los diferentes usuarios puedan verlo, interesarse y contribuir a su desarrollo final.


\section{Interdependencia de los Objetivos}
El objetivo fundamental y que se debe conseguir con este proyecto es el \textbf{OBJ-1} ya que es el desarrollo del prototipo de cliente Web basado en INDI.

Todos los objetivos planteados tienen entidad propia e independencia en sí mismo, exceptuando los objetivos secundarios \textbf{OBJ-S-1} y \textbf{OBJ-S-2}, de relevante importancia, puesto que seguir los estándares hará que se optimice notablemente el funcionamiento del prototipo.
