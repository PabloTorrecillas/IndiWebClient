\chapter{Objetivos}
\title{Objetivos}
\label{cap:Objetivos}

El objetivo fundamental de este proyecto es desarrollar un prototipo de cliente web basado en INDI que sea capaz de realizar la función control sobre un observatorio astronómico. Dicho prototipo será multiplataforma y utilizará tecnologías de carácter Software Libre. Se pretende además, que en la medida de lo posible se añadan nuevas funcionalidades en un futuro y que se convierta en un cliente web y no en un prototipo.


\section{Objetivos Principales y Objetivos Secundarios}
A continuación se detallan los objetivos principales del prototipo del cliente web:

\begin{itemize}
  \item OBJ-1: Crear un prototipo de cliente que sea capaz de realizar el control de cualquier dispositivo INDI sin apreciar su naturaleza.
  \item OBJ-2: Permitir la conexión con un servidor concreto mediante su IP y su puerto.
  \item OBJ-3: Efectuar la gestión de múltiples dispositivos que estén conectados a un servidor.
  \item OBJ-4: Desarrollar un prototipo de cliente multiplataforma, que permita hacer uso del prototipo del cliente en los principales navegadores que existen en el mercado.
  \item OBJ-5: Utilizar \texit{Software Libre} en la aplicación como herramienta de trabajo.
  \item OBJ-6: Ofrecer la posibilidad de que cualquier desarrollador pueda aportar su capacidad y conocimiento al prototipo.
  \item OBJ-7: Aportar una nueva forma de acceder a los datos de un observatorio astronómico.
  \item OBJ-8: Optimizar el uso de instrumentos de astronomía facilitando una nueva herramienta de obtención de datos.
\end{itemize}

Por otra parte, se busca conseguir los siguientes objetivos secundarios:

\begin{itemize}
  \item OBJ-S-1: Adaptar el prototipo de cliente web a las propiedades estándares de INDI.
  \item OBJ-S-2: Adecuar el prototipo a los estándares de HTML, CSS, Jquery y JavaScript.
  \item OBJ-S-3: Desarrollar el prototipo de modo que funcione correctamente con el mayor número posible de dispositivos conectados al servidor.
  \item OBJ-S-4: Conseguir que la interfaz sea maximizable y minimizable independientemente del número de dispositivos conectados al servidor.
  \item OBJ-S-5: Difundir el prototipo de cliente web en la página web de INDI.
  \item OBJ-S-6: Divulgar el prototipo de cliente web en el foro de INDI
\end{itemize}

\section{Materias y Herramientas para el Desarrollo}
