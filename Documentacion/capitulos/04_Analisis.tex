\chapter{Analisis}
\title{Analisis}
\label{cap:Analisis}

En este capitulo se realizará un análisis del proyecto para poder hacer así el futuro diseño del mismo.
Se describirán las diferentes partes que se han desarrollado.

\section{Análisis de Requisitos}
El análisis de los requisitos es el conjunto de todas las tareas que formarán las distintas necesidades de un software. Todos estos requisitos se tomarán por parte tanto del cliente como del desarrollador del proyecto ya que tiene que existir una viabilidad en el proyecto.

Se realiza un análisis de requisitos para que al llegar a la fase de diseño se tenga un software óptimo, sin contradicciones o sin ambigüedades, por ejemplo.

Los requisitos se van generando a partir de las diferentes entrevista que se tienen con el cliente.

\subsection{Requisitos Funcionales}
\begin{itemize}
  \item \textbf{RF-1:} Conectarse a un servidor INDI.
  \item \textbf{RF-2:} Mostrar todos los dispositivos conectados en el servidor INDI.
  \item \textbf{RF-3:} Obtener las propiedades de los dispositivos conectados.
  \item \textbf{RF-4:} Agrupar las propiedades de los dispositivos por diferentes grupos.
  \item \textbf{RF-5:} Editar las diferentes propiedades de INDI:
  \begin{itemize}
    \item \textbf{RF-5.1:} Propiedad Text.
    \item \textbf{RF-5.2:} Propiedad Number.
    \item \textbf{RF-5.3:} Propiedad Switch.
    \item \textbf{RF-5.4:} Propiedad Blob.
    \item \textbf{RF-5.5:} Propiedad Light.
  \end{itemize}
  \item \textbf{RF-6:} Activar la recepción de Blob en cada dispositivo conectado.
  \item \textbf{RF-7:} Desactivar la recepción de Blob en cada dispositivo conectado.
\end{itemize}
