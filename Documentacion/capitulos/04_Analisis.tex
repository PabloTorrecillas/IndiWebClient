\chapter{Analisis}
\title{Analisis}
\label{cap:Analisis}

En este capitulo se realizará un análisis del proyecto para poder hacer así el futuro diseño del mismo.
Se describirán las diferentes partes que se han desarrollado.

\section{Análisis de Requisitos}
El análisis de los requisitos es el conjunto de todas las tareas que formarán las distintas necesidades de un software. Todos estos requisitos se tomarán por parte tanto del cliente como del desarrollador del proyecto ya que tiene que existir una viabilidad en el proyecto.

Se realiza un análisis de requisitos para que al llegar a la fase de diseño se tenga un software óptimo, sin contradicciones o sin ambigüedades, por ejemplo.

Los requisitos se van generando a partir de las diferentes entrevista que se tienen con el cliente.

\subsection{Requisitos Funcionales}
\begin{itemize}
  \item \textbf{RF-1:} Conectarse a un servidor INDI.
  \item \textbf{RF-2:} Mostrar todos los dispositivos conectados en el servidor INDI.
  \item \textbf{RF-3:} Obtener las propiedades de los dispositivos conectados.
  \item \textbf{RF-4:} Agrupar las propiedades de los dispositivos por diferentes grupos.
  \item \textbf{RF-5:} Editar las diferentes propiedades de INDI:
  \begin{itemize}
    \item \textbf{RF-5.1:} Propiedad Text.
    \item \textbf{RF-5.2:} Propiedad Number.
    \item \textbf{RF-5.3:} Propiedad Switch.
    \item \textbf{RF-5.4:} Propiedad Blob.
    \item \textbf{RF-5.5:} Propiedad Light.
  \end{itemize}
  \item \textbf{RF-6:} Activar la recepción de Blob en cada dispositivo conectado.
  \item \textbf{RF-7:} Desactivar la recepción de Blob en cada dispositivo conectado.
\end{itemize}

\subsection{Requisitos No Funcionales}
\begin{itemize}
  \item \textbf{RNF-1:} Utilización de licencias libres y herramientas de desarrollo del proyecto de Software Libre para publicar el proyecto como tal.
  \item \textbf{RNF-2:} El sistema solamente podrá conectarse con dispositivos que se encuentren conectados físicamente con el servidor.
  \item \textbf{RNF-3:} Solamente se mostrarán las interfaces de los dispositivos de los cuales se hayan obtenido sus propiedades mediante el fichero XML.
  \item \textbf{RNF-4:} Se deben emplear los estándares de los diferentes lenguajes.
\end{itemize}

\section{Descripción de Actores}
Los actores son los personajes que participarán en el sistema. Tendrán a su disposición todas las funciones del sistema para poder interactuar con él.

A continuación se detalla el único usuario que tendrá el sistema:\\ \\
\textbf{AC-1: Usuario}
\begin{itemize}
  \item \textbf{Descripción:} Persona que utilizará el prototipo de cliente realizando con él las distintas funcionalidades que posee.
  \item \textbf{Características:} Es el único usuario del sistema y el que llevará a cabo todas las acciones.
  \item \textbf{Relaciones:} Ninguna.
  \item \textbf{Atributos:} Ninguno.
  \item \textbf{Comentarios:} El usuario tiene la responsabilidad de hacer una buena utilización del prototipo de cliente. Deberá tener, en mayor o menor medida, conocimientos sobre la Astronomía, pues el manejo del prototipo de cliente web se ha diseñado para trabajar con él, tanto a nivel profesional como a nivel \textit{amateur}.
\end{itemize}

\section{Descripción de Casos de Uso}
Los diferentes casos de uso nos permiten conocer como los actores se desenvuelven en el sistema, es decir, las diferentes funcionalidades que tienen disponibles en el sistema.
